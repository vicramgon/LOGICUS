\begin{abstract}

El proyecto realiza un estudio de las distintas herramientas existentes para el trabajo con la Lógica Proposicional y la Lógica de Primer Orden, de forma que aborden de una forma clara, sencilla y visual muchos de los principales algoritmos para el tratamiento de dichas lógicas.

En estos términos, y ante la imposibilidad de encontrar una que cubra los requisitos expuestos, este proyecto pretende aportar una herramienta para el tratamiento de la Lógica Computacional en un ámbito académico,  posibilitando el trabajo con conjuntos de fórmulas de proposicionales y de lenguajes de primer orden (incluyendo lenguajes con igualdad), con el desarrollo de distintos algoritmos bajo el paradigma de la programación declarativa, de manera que proporcione una herramienta capaz de servir como apoyo docente tanto al profesorado como al alumnado, y a estos efectos aglutina en sus distintos componentes y modalidades la capacidad de trabajo desde un enfoque más cercano al paradigma declarativo y la notación funcional al mismo tiempo que, a través del manejo de la interfaz web, proporciona la capacidad de un uso completo y sencillo de las funcionalidades provistas.

El proyecto se centra principalmente en el desarrollo de módulos del lenguaje Elm, que permite además el trabajo multiplataforma, a través del uso de la propia consola (\textit{elm-repl}), o a través de la compilación de los propios módulos al lenguaje web \textit{javascript}, permitiendo la integración con sistemas para la creación de documentos \textit{.html} y \textit{.md} (integración con \textit{gicentre/litvis}) o el desarrollo de una interfaz web, marcada por la accesibilidad y la usabilidad.
\end{abstract}